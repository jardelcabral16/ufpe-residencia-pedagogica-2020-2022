\documentclass[a4paper, 12pt]{article}
\usepackage[utf8]{inputenc}
\usepackage[brazil]{babel}
\usepackage[top=3cm, bottom=2cm, left=3cm, right=2cm]{geometry}
\usepackage{amsmath, amsfonts, amssymb}
\setlength{\parskip}{1em}
\usepackage{graphicx}
\usepackage{float}
\usepackage{setspace}


\begin{document}

\begin{center}
\textbf{IFPE - Matemática 5 (Saneamento) - 2021.2 \\ Avaliação final \\ Professores: Ana Karoline, André Costa e Jardel Cabral \\ 08/02/2022}
\end{center}

\noindent \textbf{Orientação}: Indique seu nome e turma na folha de respostas ou no nome do arquivo enviado. 

\noindent \textbf{Observação}: Todas as respostas devem vir acompanhadas das devidas contas ou justificativas. 
\\

\noindent \textbf{Questão 1.} Os pontos \(A(-2, 5)\), \(B(-6, 1)\) e \(C(4, -1)\) são vértices de um paralelogramo. Determine as coordenadas dos possíveis pontos para o 4º vértice.
\\

\noindent \textbf{Questão 2.} Dados \(A(2, 7)\), \(B(5, 3)\) e \(C(10, 8)\). Calcule a altura do triângulo \(ABC\) em relação ao vértice \(A\). 
\\

\noindent \textbf{Questão 3.} Determine a equação da circunferência de diâmetro \(AB\), com \(A(-2, -3)\) e \(B(3, 2)\).
\\

\noindent \textbf{Questão 4.} Identifique as cônicas e faça um esboço do gráfico cujas equações se encontram abaixo. Determine as coordenadas do(s) foco(s), do centro (ou vértice, se for uma parábola), comprimentos dos eixos, reta diretriz (se for uma parábola) ou retas assíntotas (no caso das hipérboles).    

\noindent a) $2x^2 + 4y^2 - 6x + 8y + 2 = 0$

\noindent b) $2x^2 - 4y^2 - 4x + 12y + 12 = 0$
\\

\noindent \textbf{Questão 5.} Determine as coordenadas dos focos e as equações das retas assíntotas da hipérbole de equação:
$$(x - 2)^2 - 3(y - 1)^2 = 300$$
\\

\noindent \textbf{Questão 6.} Calcule a distância entre as retas \(r: 2x - y - 8 = 0\) e \(s: 2x - y + 2 = 0\).
\\

\noindent \textbf{Questão 7.} Seja \(r\) a reta que passa por \((1, 2)\) e é perpendicular à reta \(3x + 6y - 4 = 0\). A distância de \(r\) ao centro da circunferência \(x^2 + y^2 + 2x + 8y + 13 = 0\) é:

 
\end{document}